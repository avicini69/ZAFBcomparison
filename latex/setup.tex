\documentstyle[12pt]{article}

\newcommand{\be}{\begin{equation}}
\newcommand{\ee}{\end{equation}}
\newcommand{\bea}{\begin{eqnarray}}
\newcommand{\eea}{\end{eqnarray}}
\newcommand{\smallw}{{\scriptscriptstyle W}}
\newcommand{\smallz}{{\scriptscriptstyle Z}}
\newcommand{\mt}{m_t} 
\newcommand{\mh}{m_H} 
\newcommand{\mw}{m_\smallw} 
\newcommand{\mz}{m_\smallz} 
\newcommand{\mzsq}{m_\smallz^2} 
\newcommand{\oa}{${\cal O}(\alpha)~$} 
\newcommand{\sineffl}{\sin\theta_{eff}^{\ell}\,}
\newcommand{\coseffl}{\cos\theta_{eff}^{\ell}\,}
\newcommand{\seffl}{\sin^2\theta_{eff}^{\ell}\,}
\newcommand{\ceffl}{\cos^2\theta_{eff}^{\ell}\,}

\setlength{\textwidth}{17.5cm}
\setlength{\textheight}{23cm}
\setlength{\topmargin}{-1cm}
\setlength{\oddsidemargin}{-0.5cm}
\setlength{\evensidemargin}{-0.5cm}



\begin{document}

{\Large \bf Setup of the comparisons of NC-DY codes}\\

 {\bf Input parameters}\\


\begin{tabular}{llll}
Electroweak parameters & &\\  
  $\alpha$                 = 0.007297353d0 &
  $G_\mu$                   = 1.1663787d-5  &
  $\alpha_s(\mzsq)$        = 0.1201789d0 &\\
  &~~~~~~~ (1.166389d-05 ?) & & \\
  $\mw$                    = 80.385d0 &
  $\mz$                    = 91.1876d0 & 
  $\Gamma_\smallw$           = 2.085d0 & 
  $\Gamma_\smallz$           = 2.4952d0  \\
  $\mh$                    = 125d0 & & & \\
  shifted  masses and widths & &\\
  $\mw$                    = 80.35797d0 &
   $\mz$                    = 91.15348d0 & 
  $\Gamma_\smallw$          = 2.085d0 & 
   $\Gamma_\smallz$          = 2.494266d0  \\
     fermion masses & & & \\
  $m_{\nu_e}$        = 1d-10 &
  $m_{\nu_\mu}$       = 1d-10 &
  $m_{\nu_\tau}$      = 1d-10 &  \\
  $m_e$             = 5.1099907d-4 &
  $m_\mu$           = 0.1056583d0 &
  $m_\tau$          = 1.77705d0 & \\
  $m_d$                   = 0.06983d0 &
  $m_u$                   = 0.06984d0 &
  $m_s$                   = 0.15d0 & \\
  $m_c$                   = 1.2d0 &
  $m_b$                   = 4.7d0 &
  $m_t$                   = 173d0 & \\
  conhc                    = 389379323d0  & & & \\
  ~~~~~~~~~~~(389379372.1 ?) & & & \\
\end{tabular}
%% \begin{tabular}{lll}
%%   $ \Delta \alpha (91.15348)$
%%   & POWHEG\_ew
%%   &  $\!\!\!$(5.89760567146062550E-002,-1.62163027982058471E-002) \\
%%   &  MCSANC
%%   &  5.8975530055977214E-002 (no top loop) \\
%%   &  WZGRAD
%%   &  5.8975530126199806E-002 (no top loop) \\
%%   &
%%   &  5.8916735546098963E-002 (with top loop) \\
%%   $1/\alpha(91.15348)$
%%   & POWHEG\_ew
%%   & $\!\!\!$(128.95414861873800,2.2222171242374937) \\
%%   & MCSANC
%%   & 128.95422078992519 (no top loop)       \\
%%   & WZGRAD
%%   & 128.95422078030214 (no top loop)  \\
%%   &
%%   & 128.96227776336096 (with top loop) \\
%%   $\Delta r$
%%   &  POWHEG\_ew
%%   &  $\!\!\!$(2.97632672697318683E-002,-2.89767823517196148E-002) \\
%%   & MCSANC
%%   & 2.9762759543028511E-002\\                                                   
%%   & WZGRAD
%%   & 2.9762761199628920E-002
%% \end{tabular}
\begin{tabular}{lll}
  $ \Delta \alpha (91.15348)$
  & POWHEG\_ew
  &  $\!\!\!$(5.8976057E-002,-1.6216303E-002) \\
  &  MCSANC
  &  5.8975530E-002 (no top loop) \\
  &  WZGRAD
  &  5.8975530E-002 (no top loop) \\
  &
  &  5.8916736E-002 (with top loop) \\
  $1/\alpha(91.15348)$
  & POWHEG\_ew
  & $\!\!\!$(128.95415,2.2222171) \\
  & MCSANC
  & 128.95422 (no top loop)       \\
  & WZGRAD
  & 128.95422 (no top loop)  \\
  &
  & 128.96228 (with top loop) \\
  $\Delta r$
  &  POWHEG\_ew
  &  $\!\!\!$(2.9763267E-002,-2.8976782E-002) \\
  & MCSANC
  & 2.9762759E-002\\                                                   
  & WZGRAD
  & 2.9762761E-002
\end{tabular}

\vskip0.5cm



{\bf Additional choices}\\
Final-state: bare muons.\\
The choice of the proton PDF set is currently {\tt MSTW2008nlo68cl}.
A consistent discussion of the photon-induced processes would require the choice
of {\tt NNPDF3.1} or of {\tt MMHT2015}, which feature both a QCD-only and a full QCD+QED DGLAP evolution of the parton densities.\\
The factorisation and renormalisation scales are set equal to the lepton-pair
invariant mass $\mu_R=\mu_F=M_{ll}$.\\
The EW input schemes for all the technical comparisons are:
(i) $(\alpha,\mw,\mz)$, (ii) $(G_\mu,\mw,\mz)$.
When possible,
also (iii) $(\alpha,\seffl,\mz)$ and (iv) $(G_\mu,\seffl,\mz)$
should be considered.
Hybrid schemes, using different couplings
to parameterise the QED and weak corrections,
are part of the discussion about the best predictions.\\
Different prescriptions available to define
the pole of the gauge boson propagators
should be part of the discussion of the higher-order effects and uncertainties.\\[3mm]



{\bf Observables and cuts}\\
Cuts: $M_{ll}> 50$ GeV in all the simulations.\\
The invariant mass distributions are studied in the range
$M_{ll}\in [60,150]$ GeV with a bin size of 1 GeV.
The cross sections integrated in specific mass windows can be obtained by combining the differential results.

Two different sets of additional cuts are considered:
(a) fully inclusive acceptance and
(b) standard fiducial cuts on the leptons, i.e.
$p_\perp^{l^+}>25$ GeV,
$p_\perp^{l^-}>25$ GeV,
$|\eta_{l^+}|<2.5$,
$|\eta_{l^-}|<2.5$.
The LHCb configuration with $2<\eta_{l^\pm}<4.5$ could be of interest.\\[5mm]
Observables: total, forward and backward invariant mass distributions.
The definition of forward (backward) relies on the value of the scattering angle
of the negatively charged lepton, in the Collins-Soper frame.\\
Shall we include also $A_4$ ?\\[5mm]

{\bf Approximations}\\
We consider the LO and NLO-EW results for a technical comparison
(i.e. where we ideally expect to obtain identical results by all the codes ).\\
We break the NLO-EW results into QED (ISR,IFI,FSR) and pure weak (PW) subsets.\\
We consider for high-order (HO) contributions:
(i) the universal corrections which can be reabsorbed into a redefinition
of the lowest order couplings;
(ii) higher-order QED effects;
(iii) effects due to the matching between NLO-EW and multiple photon effects.



\end{document}





